\documentclass[a4paper,10pt]{paper}
\usepackage[utf8]{inputenc}
\usepackage{listings}
\usepackage{minted}
\usepackage{color}
\usepackage{graphicx}
\usepackage{amsmath}
\usepackage{textcomp}
\usepackage{geometry}
 \geometry{
 a4paper,
 total={170mm,257mm},
 left=20mm,
 top=20mm,
 }

\definecolor{mygreen}{rgb}{0,0.6,0}
\definecolor{mygray}{rgb}{0.5,0.5,0.5}
\definecolor{mymauve}{rgb}{0.58,0,0.82}


% Title Page
\title{Informe F\'isica Computacional IV}
\author{Bruno Sanchez}


\begin{document}

\maketitle

\begin{abstract}
Se presentan los enunciados, soluciones y discusiones de resultados del cuarto trabajo pr\'actico
de la materia de F\'{i}sica Computacional 2016.
\end{abstract}

\section{Introducci\'on}
\subsection{Ensamble can\'onico}
El \'area de la mec\'anica estad\'istica se caracteriza por atacar problemas de microf\'isica a trav\'es
de una perspectiva macrosc\'opica; esto es, prediciendo comportamientos de variables
macrosc\'opicas a traves de modelos microsc\'opicos.
Esto es posible mediante el c\'alculo de los llamados \textit{valores medios}, mediante la teor\'ia de \textit{ensambles}.

Esta supone un conjunto de sistemas que se han preparado de forma id\'entica, y que poseen alg\'un \textit{microestado}
posible, pero un mismo \textit{macroestado} determinado.

Las propiedades macrosc\'opicas del sistema estan determinadas por los valores medios sobre el ensable.
Existen diferentes tipos de ensambles para distintos sistemas.
En particular se centrar\'a el estudio en el \textit{Ensamble Can\'onico} el cual sirve para sistemas que 
posee el n''umero de part\'iculas $N$, el valor de temperatura $T$ y el volumen $V$ constante.
Esto es equivalente a decir que el sistema esta en contacto, o que intercambia energ\'ia, con un 
\textit{reservorio t\'ermico}.

\subsection{El modelo de Ising}
El modelo de Ising es una de los modelos m\'as importantes de la mecanica estad\'istica. Propuesto
en 1960 por W. Lenz, como un modelo b\'asico del problema de la magnetizaci\'on espont\'anea
en los materiales ferromagn\'eticos. El mismo plantea un arreglo de dipolos (o \textit{spines}) que interact\'uan
\'unicamente con sus primeros vecinos, de forma tal que la energ\'ia del sistema, en ausencia de 
campos externos esta dada por:
\begin{displaymath}
 E = - \sum\limits_{i\neq j}J_{i,j} s_i s_j 
\end{displaymath}
 donde los valores de $s_k$ pueden ser $\pm1$, y la constante $J$ determina el tipo de interacci\'on.
 En nuestro caso $J>0$, lo cual indica interacc\'on ferromagn\'etica y el arreglo de spines a utilizar ser\'a cuadrado.
 Las orientaciones de los dipolos a lo largo de la red definen cuanta energ\'ia tiene el sistema, 
 y cada cambio que ocurre en un dado dipolo afecta esta cantidad global.

 Las cantidades macrosc\'opicas a calcular ser\'an:
 \begin{itemize}
  \item Magnetizaci\'on: $\bar{m} = \frac{\mu}{N} \sum\limits_{i=1}^{N} s_i$
  \item Calor espec\'ifico: $C_v = \frac{1}{k_B T^2}(\overline{E^2} -\overline{E}^2)$
  \item Susceptibilidad magn\'etica: $\chi = \frac{1}{k_B T^2}(\overline{m^2} -\overline{m}^2)$
 \end{itemize}



\end{document}          
