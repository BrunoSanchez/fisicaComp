\documentclass[a4paper,10pt]{paper}
\usepackage[utf8]{inputenc}
\usepackage{listings}
\usepackage{minted}
\usepackage{color}
\usepackage{graphicx}
\usepackage{amsmath}
\usepackage{textcomp}
\usepackage{geometry}
 \geometry{
 a4paper,
 total={170mm,257mm},
 left=20mm,
 top=20mm,
 }

\definecolor{mygreen}{rgb}{0,0.6,0}
\definecolor{mygray}{rgb}{0.5,0.5,0.5}
\definecolor{mymauve}{rgb}{0.58,0,0.82}


% Title Page
\title{Informe F\'isica Computacional IV}
\author{Bruno Sanchez}


\begin{document}

\maketitle

\begin{abstract}
Se presentan los enunciados, soluciones y discusiones de resultados del cuarto trabajo pr\'actico
de la materia de F\'{i}sica Computacional 2016.
\end{abstract}

\section{Introducci\'on}
\subsection{Ensamble can\'onico}
El \'area de la mec\'anica estad\'istica se caracteriza por atacar problemas de microf\'isica a trav\'es
de una perspectiva macrosc\'opica; esto es, prediciendo comportamientos de variables
macrosc\'opicas a traves de modelos microsc\'opicos.
Esto es posible mediante el c\'alculo de los llamados \textit{valores medios}, mediante la teor\'ia de \textit{ensambles}.

Esta supone un conjunto de sistemas que se han preparado de forma id\'entica, y que poseen alg\'un \textit{microestado}
posible, pero un mismo \textit{macroestado} determinado.

Las propiedades macrosc\'opicas del sistema estan determinadas por los valores medios sobre el ensable.
Existen diferentes tipos de ensambles para distintos sistemas.
En particular se centrar\'a el estudio en el \textit{Ensamble Can\'onico} el cual sirve para sistemas que 
posee el n''umero de part\'iculas $N$, el valor de temperatura $T$ y el volumen $V$ constante.
Esto es equivalente a decir que el sistema esta en contacto, o que intercambia energ\'ia, con un 
\textit{reservorio t\'ermico}.

\subsection{El modelo de Ising}
El modelo de Ising es una de los modelos m\'as importantes de la mecanica estad\'istica. Propuesto
en 1960 por W. Lenz, como un modelo b\'asico del problema de la magnetizaci\'on espont\'anea
en los materiales ferromagn\'eticos. El mismo plantea un arreglo de dipolos (o \textit{spines}) que interact\'uan
\'unicamente con sus primeros vecinos, de forma tal que la energ\'ia del sistema, en ausencia de 
campos externos esta dada por:
\begin{displaymath}
 E = - \sum\limits_{i\neq j}J_{i,j} s_i s_j 
\end{displaymath}
 donde los valores de $s_k$ pueden ser $\pm1$, y la constante $J$ determina el tipo de interacci\'on.
 En nuestro caso $J>0$, lo cual indica interacc\'on ferromagn\'etica y el arreglo de spines a utilizar ser\'a cuadrado.
 Las orientaciones de los dipolos a lo largo de la red definen cuanta energ\'ia tiene el sistema, 
 y cada cambio que ocurre en un dado dipolo afecta esta cantidad global.

 Las cantidades macrosc\'opicas a calcular ser\'an:
 \begin{itemize}
  \item Magnetizaci\'on: $\bar{m} = \frac{\mu}{N} \sum\limits_{i=1}^{N} s_i$
  \item Calor espec\'ifico: $C_v = \frac{1}{k_B T^2}(\overline{E^2} -\overline{E}^2)$
  \item Susceptibilidad magn\'etica: $\chi = \frac{1}{k_B T^2}(\overline{m^2} -\overline{m}^2)$
 \end{itemize}

 \subsection{Funci\'on de autocorrelaci\'on}
 La funci\'on de autocorrelaci\'on es un estad\'istico importante para analizar la evolucion de las 
 series temporales (o cualquier tipo de serie), ya que es capaz de cuantificar la dependencia de la
 cantidad a estudiar con su historia. 
 Esto es importante por ejemplo para conocer si, en caso de nuestras simulaciones de Monte Carlo, una
 condici\'on inicial de simulaci\'on, artificialmente impuesta, impactar\'a en el resultado ulterior, 
 y en caso de que lo haga, cuanto dura esta influencia.
 La definici\'on de la funci\'on de correlaci\'on es similar a la integral de convoluci\'on de dos funciones:
\begin{displaymath} 
 (f \star g)(u) = \int\limits_{-\infty}^{\infty}f(x)g(x+u)dx
\end{displaymath}
 con la diferencia clave, en la que el signo de la variable de desplazamiento $u$ (llamada tambien \textit{delay})
 es positivo.
 La funcion de \textbf{auto}correlaci\'on se define como la correlaci\'on de una funci\'on consigo misma:
 \begin{displaymath} 
 A_f(u) = (f \star f)(u) = \int\limits_{-\infty}^{\infty}f(x)f(x+u)dx
\end{displaymath}
 y en casos discretos, tal como en simulaciones num\'ericas, existe la posibilidad de estimarla mediante la 
 siguiente f\'ormula normalizada:
 \begin{displaymath} 
 A_f(k) = \frac{\langle f_i f_{i+k} \rangle - \langle f_i\rangle^2}{\langle f_i^2 \rangle - \langle f_i\rangle^2}
\end{displaymath}
 la cual posee la propiedad de que $A(0)=1$. 
 
\section{Ejercicio a}
Se pide calcular la energ\'ia y la magnetizaci\'on en funcion de los pasos de montecarlo para redes de tama\~nos
10, 20 y 40. Se utiliz\'o una temperatura del sistema de un valor $T = 1.01\times T_c$.
Como se puede apreciar en la figura \ref{fig:eja_e_vs_t} la energ\'ia de los sistemas es independiente
del tama\~no (se puede ver que las redes de mayor tama\~no se grafican m\'as arriba), y muy cercana a cero.
\begin{figure}
 \centering
 \includegraphics[width=0.7\textwidth,keepaspectratio=true]{../graficos/e_vs_t.png}
 % e_vs_t.png: 0x0 pixel, 300dpi, 0.00x0.00 cm, bb=
 \caption{Energ\'ia en funci\'on de pasos de montecarlo. Se adicionaron constantes para diferenciar los resultados para distintos tama\~nos.}
 \label{fig:eja_e_vs_t}
\end{figure}

En cuanto a la magnetizaci\'on se hallaron los resultados de la figura \ref{fig:eja_m_vs_t}, donde se pueden ver en paneles diferentes
la evolucion de la magnetizaci\'on a lo largo de una porci\'on de la simulaci\'on montecarlo.

\begin{figure}
 \centering
 \includegraphics[width=0.7\textwidth]{../graficos/m_vs_t.png}
 % m_vs_t.png: 0x0 pixel, 300dpi, 0.00x0.00 cm, bb=
 \caption{Magnetizaci\'on en funcion de paso de montecarlo para las diferentes redes.}
 \label{fig:eja_m_vs_t}
\end{figure}

Para estudiar mejor estas cantidades se calcularon los valores de la funcion de autocorrelaci\'on para la energ\'ia y la 
magnetizaci\'on.
\begin{figure}
 \centering
 \includegraphics[width=0.7\textwidth,keepaspectratio=true]{../graficos/e_corr.png}
 % e_corr.png: 0x0 pixel, 300dpi, 0.00x0.00 cm, bb=
 \caption{Funci\'on de autocorrelaci\'on para la energ\'ia, en diferentes redes.}
 \label{fig:e_corr}
\end{figure}

\begin{figure}
 \centering
 \includegraphics[width=0.7\textwidth,keepaspectratio=true]{../graficos/m_corr.png}
 % e_corr.png: 0x0 pixel, 300dpi, 0.00x0.00 cm, bb=
 \caption{Funci\'on de autocorrelaci\'on para la magnetizaci\'on, en diferentes redes.}
 \label{fig:e_corr}
\end{figure}

\section{Ejercicio b}
Se pide estudiar las dependencias de las cantidades magnetizaci\'on, energ\'ia, calor espec\'ifico, y
susceptibilidad magn\'etica con la temperatura. Para esto se comenzo a temperaturas por encima de la
temperatura cr\'itica, y se descendio hasta valores cercanos a cero.



\end{document}          
