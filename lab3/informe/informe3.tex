\documentclass[a4paper,10pt]{paper}
\usepackage[utf8]{inputenc}
\usepackage{listings}
\usepackage{minted}
\usepackage{color}
\usepackage{graphicx}
\usepackage{amsmath}
\usepackage{textcomp}
\usepackage{geometry}
 \geometry{
 a4paper,
 total={170mm,257mm},
 left=20mm,
 top=20mm,
 }

\definecolor{mygreen}{rgb}{0,0.6,0}
\definecolor{mygray}{rgb}{0.5,0.5,0.5}
\definecolor{mymauve}{rgb}{0.58,0,0.82}


% Title Page
\title{Informe F\'isica Computacional III}
\author{Bruno Sanchez}


\begin{document}

\maketitle

\begin{abstract}
Se presentan los enunciados, soluciones y discusiones de resultados del tercer trabajo pr\'actico
de la materia de F\'{i}sica Computacional 2016.
\end{abstract}


\section{Ejercicio 2: Caminatas al azar}
\subsection{Enunciado}
 En una red cuadrada bidimiensional implementar un algoritmo de caminatas al azar de $n$ pasos. Iniciar la
 caminata en el sitio central de la red $R(0) = (0,0)$.
 \begin{enumerate}
  \item Hallar el desplazamiento cuadr\'atico medio 
  $\langle(R(n) - R(0))^2\rangle$ en funci\'on del n\'umero
  de pasos $n$, promediando sobre $N=10^6$ caminatas. 
  Chequear que se cumpla la ley $\langle R^2\rangle\sim n$.
  \item Subdividr la red en cuadrantes y contabilizar la cantidad de veces que el caminante termina
  en un dado cuadrante, comparando con el valor esperado $N/4$. Graficar estos resultados en funci\'on
  de $n$, comparando los resultados de distintos generadores. 
  Utilizar los generadores \textbf{ran2}, \textbf{MZRAN} y el llamado ``Mersenne Twister''.
  \item Realizar el mismo ejercicio que en el primero, pero para una part\'icula real en 3-D.
 \end{enumerate}

\subsection{Soluci\'on}
 Para la resoluci\'on del problema se escribi\'o un programa en Fortran 90: 
 \inputminted[firstline=16, linenos, firstnumber=1]{fortran}{../ej2ab.f90}
 
 En la l\'inea 31 el programa sortea n\'umeros \textit{random} entre los valores $(0, 1)$.
 Y la funci\'on \texttt{step} se utiliza este mismo valor \textit{random}, para decidir que direcci\'on
 de movimiento tomar\'a la caminata. En particular la funci\'on \texttt{step} discretiza el paso, de
 forma tal que el caminante da un paso de largo 1 hacia arriba, abajo, izquierda o derecha.
 
 En la l\'inea 35 se acumulan los valores de $\Delta R^2$ para el c\'alculo de $\langle R^2\rangle$.
 Y en la l\'inea posterior (36) se llama a la \texttt{subroutine quadrant}, que acumula el conteo sobre los
 cuadrantes.

\subsection{Discusi\'on}
En la figura \ref{fig:ej2} se muestra un gr\'afico del error en relaci\'on a la cantidad de
pasos $n$. Se corrobora el comportamiento lineal del desplazamiento medio para cualquiera de los
tres generadores \textit{random} utilizados. Para poder realizar una comparaci\'on, se adicion\'o
un valor de 10 y 20 a los valores calculados utilizando dos de los generadores.

\begin{figure}[H]
 \centering
 \includegraphics[width=0.7\textwidth]{../graficos/rms_vs_n.png}
 % ej_2.png: 0x0 pixel, 300dpi, 0.00x0.00 cm, bb=
 \caption{Error vs cantidad de pasos $n$}
 \label{fig:ej2}
\end{figure}

En las figuras \ref{fig:ej2quadsranmod}, \ref{fig:ej2quadsmzran} y \ref{fig:ej2quadsmstw} se 
muestran gr\'aficos los contadores del cuadrante final de las caminatas para los generadores
\texttt{ran2}, \texttt{mzran} y \texttt{mersenne twister} respectivamente.
En los paneles inferiores de las figuras se muestra la cantidad de veces que la caminata termina
en el origen en funci\'on del n\'umero de pasos; demostrando claramente que la probabilidad
de que esto ocurra es distinta de cero. 

\begin{figure}[H]
 \centering
 \includegraphics[width=0.7\textwidth,keepaspectratio=true]{../graficos/quads_vs_nranmod.png}
 % quads_hist_ranmod.png: 0x0 pixel, 300dpi, 0.00x0.00 cm, bb=
 \caption{Contadores de cuadrantes para el generador \texttt{ran2}.}
 \label{fig:ej2quadsranmod}
\end{figure}

\begin{figure}[H]
 \centering
 \includegraphics[width=0.7\textwidth,keepaspectratio=true]{../graficos/quads_vs_n_mzran.png}
 % quads_hist_ranmod.png: 0x0 pixel, 300dpi, 0.00x0.00 cm, bb=
 \caption{Contadores de cuadrantes para el generador \texttt{mzran}}
 \label{fig:ej2quadsmzran}
\end{figure}

\begin{figure}[H]
 \centering
 \includegraphics[width=0.7\textwidth,keepaspectratio=true]{../graficos/quads_vs_n_mstw.png}
 % quads_hist_ranmod.png: 0x0 pixel, 300dpi, 0.00x0.00 cm, bb=
 \caption{Contadores de cuadrantes para el generador \texttt{mersennetwister}}
 \label{fig:ej2quadsmstw}
\end{figure}


\section{Ejercicio 3: Integraci\'on por Monte Carlo}
Para este ejercicio es necesario aplicar el M\'etodo de Monte Carlo para integraci\'on.

\subsubsection{Enunciado}
\begin{enumerate}
 \item Escribir un programa  que  estime  la  integral  de  la  funci\'on $f(x)  =x^n$ en  el intervalo 
 $[0;1]$ usando el m\'etodo de Monte Carlo. Para el caso $n= 3$, 
 estime la integral usando $N=10\times i$; $i= 1, 2, 3 \ldots $
evaluaciones de la func\'on. Calcule la diferencia entre el valor exacto de integral, 
$I= 1=(n+ 1)$, y el obtenido usando el m\'etodo de Monte Carlo, 
y grafique que el valor absoluto de \'esta en funci\'on de $N$. 
  \item Modificar el programa anterior para calcular la misma integral, pero ahora utilizando
  \textit{importance sampling}, con la distribuci\'on de probabilidad $p(x) = (k+1) x^k$, con $k<n$. 
  Recordar el m\'etodo de la transformada inversa para obtener n\'umeros aleatorios con distribuci\'on
  no uniforme.
\end{enumerate}
 
\subsubsection{Soluci\'on}
Se program\'o el siguiente c\'odigo, que resolvia el calculo num\'erico del ejercicio, para el primer
apartado:
\inputminted[firstline=16, linenos, firstnumber=1]{fortran}{../ej3a.f90}

Para el siguiente apartado se program\'o el c\'odigo a continuaci\'on:
\inputminted[firstline=16, linenos, firstnumber=1]{fortran}{../ej3b.f90}


\subsubsection{Discusi\'on}
Para el ejercicio se utiliz\'o el generador de n\'umeros \textit{random} \texttt{Mersenne twister}, 
y el llamado \texttt{ran2} (en el segundo apartado).
En el primer caso, se utiliza una estimaci\'on del valor de la integral mediante un Monte Carlo 
cl\'asico. En el segundo apartado se puede observar la presencia de ambas funciones \texttt{p(x, k)} y 
\texttt{phi\_inv(x, k)} las cuales son las funciones densidad de probabilidad y inversa de funci\'on 
acumulada de probabilidad, respectivamente.
El m\'etodo para sortear n\'umeros random con distribuci\'on no uniforme es el de la transformada inversa:
se utiliza una transformaci\'on del intervalo $(0,1)$ mediante la funci\'on inversa de la distribuci\'on 
acumulada de probabilidad.
\begin{displaymath}
 \phi(X) = p(X>x) = \int_{-\infty}^{X}p(x)dx
\end{displaymath}
Luego se realiza la inversi\'on de esta funci\'on, y se eval\'ua en el n\'umero aleatorio $r$
\begin{displaymath}
 y = \phi^{-1}(r)
\end{displaymath}

Para nuestro caso, la inversa $\phi^{-1}$ es tal que $y = x^{\frac{1}{k+1}}$, y $k=2$ y $k=3$.

En la figura \ref{fig:ej3ab} se muestra el gr\'afico del error en funci\'on del n\'umero de sorteos
de Monte Carlo utilizados.
Es notable que con el \textit{importance sampling} el c\'alculo mejora notablemente para $k=2$, y 
excepcionalmente para $k=3$, ya que no tiene error a precisi\'on de m\'aquina.
Esto se debe que la distribuci\'on de probabilidad es una constante por la funci\'on a integrar.

\begin{figure}
 \centering
 \includegraphics[width=0.7\textwidth]{../graficos/errorMC.png}
 % errorMC.png: 0x0 pixel, 300dpi, 0.00x0.00 cm, bb=
 \caption{Gr\'afico de error en funci\'on del n\'umero de sorteos aleatorios usados para estimar la 
 integral.}
 \label{fig:ej3ab}
\end{figure}


\end{document}          
