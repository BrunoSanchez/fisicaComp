\documentclass[a4paper,10pt]{report}
\usepackage[utf8]{inputenc}
\usepackage{listings}
\usepackage{minted}
\usepackage{color}

\definecolor{mygreen}{rgb}{0,0.6,0}
\definecolor{mygray}{rgb}{0.5,0.5,0.5}
\definecolor{mymauve}{rgb}{0.58,0,0.82}


% Title Page
\title{Informe F\'isica Computacional }
\author{Bruno Sanchez}


\begin{document}
\maketitle

\begin{abstract}
Se presentan los enunciados, soluciones y discusion de resultados del primer trabajo practico
de la materia de fisica Computacional.
\end{abstract}

\section{Ejercicio 1}

\section{Ejercicio 2}

\subsection{Enunciado}
Sea una funci\'on $F(x) = e^x$. Evaluar $F'(x=1)$ mediante la formula centrada de dos puntos 
 \begin{displaymath}
 f'(x) = \frac{f(x+h) - f(x-h)}{2h} + O(h^2)
 \end{displaymath} 
 para distintos valores de $h$ y calcule el incremento \'optimo teniendo en cuenta los errores de truncamiento y redondeo. 
 Grafique el error (usando el valor exacto de la derivada) versus $h$ (elija $h=10^{-k}$, con $k$ entero y grafique usando
 escala $log-log$.
 
\subsection{Soluci\'on}
Como primer paso se calculo la derivada de la funcion $F$, la cual es $F'(x)=e^x$, la cual evaluada en $x=1$ entrega un
valor de $e$. 

Luego se escribi\'o un programa en Fortran 90, para calcular la f\'ormula centrada de dos puntos:
\inputminted[firstline=22]{fortran}{../2/lab1_2.f90}


\end{document}          
