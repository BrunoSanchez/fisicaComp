\documentclass[a4paper,10pt]{paper}
\usepackage[utf8]{inputenc}
\usepackage{listings}
\usepackage{minted}
\usepackage{color}
\usepackage{graphicx}
\usepackage{amsmath}
\usepackage{textcomp}
\usepackage{geometry}
 \geometry{
 a4paper,
 total={170mm,257mm},
 left=20mm,
 top=20mm,
 }

\definecolor{mygreen}{rgb}{0,0.6,0}
\definecolor{mygray}{rgb}{0.5,0.5,0.5}
\definecolor{mymauve}{rgb}{0.58,0,0.82}


% Title Page
\title{Informe F\'isica Computacional }
\author{Bruno Sanchez}


\begin{document}

\maketitle

\begin{abstract}
Se presentan los enunciados, soluciones y discusiones de resultados del primer trabajo pr\'actico
de la materia de F\'{i}sica Computacional 2016.
\end{abstract}

\section{Ejercicio 2}
\subsection{Enunciado}
Sea una funci\'on $F(x) = e^x$. Evaluar $F'(x=1)$ mediante la formula centrada de dos puntos 
 \begin{displaymath}
 f'(x) = \frac{f(x+h) - f(x-h)}{2h} + O(h^2)
 \end{displaymath} 
 para distintos valores de $h$ y calcule el incremento \'optimo teniendo en cuenta los errores 
 de truncamiento y redondeo. 
 Grafique el error (usando el valor exacto de la derivada) versus $h$ (elija $h=10^{-k}$, 
 con $k$ entero y grafique usando escala $log-log$.
 
\subsection{Soluci\'on}
Como primer paso se calculo la derivada de la funcion $F$, la cual es $F'(x)=e^x$, 
la cual evaluada en $x=1$ entrega un valor de $e$. 

Luego se escribi\'o un programa en Fortran 90, para calcular la f\'ormula centrada de dos puntos:
\inputminted[firstline=22]{fortran}{../2/lab1_2.f90}

\subsection{Discusi\'on}
En la figura \ref{fig:ej2} se muestra un gr\'afico del error en relaci\'on al tama\~no del paso $h$. Se observa como esto
es una funci\'on decreciente hasta cierto $h$ donde vuelve a crecer. Esto es debido a la precisi\'on de la 
m\'aquina, la cual al efectuar restas entre valores cercanos comete imprecisiones de redondeo cada vez mayores.
Esto sucede dentro del argumento de la funci\'on $f$, donde al sumar y restar $x+h$ y $x-h$, con un $h$ tan peque\~no, 
provocan que ambas evaluaciones de $f$ sean iguales a \'orden de precisi\'on de m\'aquina. Con esto, la resta del numerador
del cociente incremental posee un error de redondeo que puede ser cercano al 100\textdiscount.
\begin{figure}[H]
 \centering
 \includegraphics[width=0.6\textwidth]{../graficos/ej_2.png}
 % ej_2.png: 0x0 pixel, 300dpi, 0.00x0.00 cm, bb=
 \caption{Error vs tama\~no de paso $h$}
 \label{fig:ej2}
\end{figure}


\section{Ejercicio 8: Sistemas Ca\'oticos}
Para el ejercicio 8, se debe considerar el siguiente mapeo log\'{i}stico:
\begin{displaymath}
 x_{t+1} = r x_t (1 - x_t)
\end{displaymath}
para $k \in N$.
\subsection{Apartado a}
\subsubsection{Enunciado}
Se solicita graficar las trayectorias para $x_t$ vs. $t$, para $r = 1.5; 3.3; 3.5; 3.55; 4$. 
Es necesario realizar 500 pasos para diferentes condiciones iniciales $x_0 = 0.06; 0.3; 0.6; 0.9$, 
y graficar las mismas.
\subsubsection{Soluci\'on}
Se program\'o el siguiente c\'odigo, que resolvia el calculo num\'erico del ejercicio:
\inputminted[firstline=16, linenos, firstnumber=1]{fortran}{../8/ej8a.f90}
\subsubsection{Discusi\'on}
El gr\'afico de la trayectoria para $r = 1.5$ se puede apreciar en la figura \ref{fig:ej8a_r_150}, 
y donde es facilmente observable una convergencia en pocos pasos a un valor estable.
Para los valores superiores de $r$, este comportamiento no se observa.
Los gr\'aficos de las figuras \ref{fig:ej8a_r_330}, \ref{fig:ej8a_r_350}, y \ref{fig:ej8a_r_355} muestran comportamiento
oscilatorio para $t$ mayores que un cierto valor; y ninguna progresi\'on converge a un valor estacionario, sino que 
los mapeos muestran un grupo finito de valores posibles para $x_t$, entre los cuales va oscilando.
\begin{figure}[H]
 \centering
 \includegraphics[width=0.6\textwidth,keepaspectratio=true]{../graficos/ej8a_r_150.png}
 % ej8a_r_1.50.png: 0x0 pixel, 300dpi, 0.00x0.00 cm, bb=
 \caption{Trayectorias para diferentes condiciones iniciales, con $r=1.5$ y $t<20$}
 \label{fig:ej8a_r_150}
\end{figure}
\begin{figure}[H]
 \centering
 \includegraphics[width=0.6\textwidth,keepaspectratio=true]{../graficos/ej8a_r_330.png}
 % ej8a_r_330.png: 0x0 pixel, 300dpi, 0.00x0.00 cm, bb=
 \caption{Trayectorias para $r=3.3$ con $t<30$}
 \label{fig:ej8a_r_330}
\end{figure}
\begin{figure}[H]
 \centering
 \includegraphics[width=0.6\textwidth,keepaspectratio=true]{../graficos/ej8a_r_350.png}
 % ej8a_r_330.png: 0x0 pixel, 300dpi, 0.00x0.00 cm, bb=
 \caption{Trayectorias para $r=3.5$ con $t<30$}
 \label{fig:ej8a_r_350}
\end{figure}
\begin{figure}[H]
 \centering
 \includegraphics[width=0.6\textwidth,keepaspectratio=true]{../graficos/ej8a_r_355.png}
 % ej8a_r_330.png: 0x0 pixel, 300dpi, 0.00x0.00 cm, bb=
 \caption{Trayectorias para $r=3.55$ con $t<30$}
 \label{fig:ej8a_r_355}
\end{figure}

Para un valor de $r=4.0$ el comportamiento del mapeo $x_t$ es totalmente diferente a los anteriormente
descriptos. Desde valores de $t$ bajos, la serie de puntos $x_t$ es en apariencia impredecible --aunque
esto no quiere decir que no podamos acotar la progresi\'on-- y posee valores de $x_t \in (0, 1)$.
El gr\'afico de la figura  \ref{fig:ej8a_r_400} muestra, tan s\'olo para los 50 primeros valores de $t$, 
la trayectoria de $x_t$, donde es evidente que no es esperable una convergencia similar a las vistas anteriormente.
Adem\'as este comportamiento no cambia para valores mayores de $t$, al menos hasta $t=500$ donde se calcularon
las trayectorias.

\begin{figure}[H]
 \centering
 \includegraphics[width=0.6\textwidth,keepaspectratio=true]{../graficos/ej8a_r_400.png}
 % ej8a_r_330.png: 0x0 pixel, 300dpi, 0.00x0.00 cm, bb=
 \caption{Trayectorias para $r=4$ con $t<50$}
 \label{fig:ej8a_r_400}
\end{figure}

\subsection{Apartado b}
\subsubsection{Enunciado}
Se solicita calcular el espectro de potencia para los valores de $r$ anteriormente mencionados, mediante 
la transformada de Fourier FFT. Como dato adicional se sabe que es necesario descartar los primeros 300 
pasos, y que se debe utilizar tan solo la condici\'on inicial $x_0 = 0.6$ 

\subsubsection{Soluci\'on}
Se program\'o el siguiente c\'odigo, que resolvia el calculo num\'erico del ejercicio:
\inputminted[firstline=16, linenos, firstnumber=1]{fortran}{../8/ej8b.f90}

\subsubsection{Discusi\'on}
Para este ejercicio fue necesario utilizar la librer\'{i}a de FFTW, para eso se requiere una primera
instancia de planeamiento de la transformada (l\'{i}nea 35), y luego de guardar los valores a transformar
(l\'{i}nea 36) se llama a la subrutina de trasformada (l\'{i}nea 37). Como la transformada de Fourier de una
serie real cumple con $\hat{f}(-k) = \hat{f}(k)^*$, y al tomar la parte real se vuelve una funci\'on par, 
el bucle de las l\'{i}neas 46-58 se encarga de guardar la parte correspondiente a las frecuencias menores que $0$.

En la figura \ref{fig:ej8b} se muestran los cuatro espectros de potencia calculados para cada valor de $r$.
Notar que se aplic\'o una normalizaci\'on de forma tal que sus valores m\'aximos sean iguales a la unidad, y de 
esta forma poder realizar un an\'alisis comparativo de la importancia relativa de las frecuencias para cada valor
de $r$. 

Es posible apreciar la r\'apida convergencia de la serie correspondiente a $r=1.3$, ya que no presenta 
se\~nal alguna en el rango de frecuencia muestreado. Esto es bastante razonable, ya que vimos que los 
valores de $x_t$ crecen y suavemente se estacionan en un valor sin oscilar.

Esto es muy distinto si observamos los espectros correspondientes a $r=3.3$, $r=3.5$ y $r=3.55$, donde hay
se\~nal en dos bandas de frecuencia, mostrando picos prominentes en frecuencias intermedias. Esto es la contracara
de las oscilaciones previamente discutidas, donde vimos que los valores de $x_t$ posibles --para un $t$ mayor que
un cierto valor-- son pocos y se repiten peri\'odicamente. 

Nos reservamos el espectro de $r=4.$ para el final, donde se ve se\~nal en varias bandas del espectro, y con un pico 
dominante en las cercanias al valor de $f=45$, mostrando una cuasi-periodicidad. Esto quiere decir que si bien, 
el comportamiento de la serie $x_t$ es impredecible, es altamente probable que los valores posibles de $x_t$ sean 
pertenecientes a un conjunto finito, dadas las condiciones iniciales y para un $r$ fijo.

\begin{figure}[!h]
 \centering
 \includegraphics[width=0.6\textwidth,keepaspectratio=true]{../graficos/ej8b.png}
 % ej8b.png: 0x0 pixel, 300dpi, 0.00x0.00 cm, bb=
  \caption{Espectros de potencia para distintos valores de $r$, con condici\'on inicial $x_0 = 0.6$ donde se han omitido
  los primeros 300 pasos ($t>300$). Los espectros de potencia estan normalizados de forma tal que su m\'aximo valor sea igual a uno.}
 \label{fig:ej8b}
\end{figure}

\subsection{Apartado c}
\subsubsection{Enunciado}
Se pide graficar un histograma normalizado de la distribuci\'on de $x_t$ para $r=4$ y $x_0=0.6$. Adem\'as se pide descartar los primeros
300 pasos, y utilizar hasta el valor de $x_{10000}$.

\subsubsection{Discusi\'on}
En la figura \ref{fig:8c} se puede ver el gr\'afico del histograma solicitado. Si bien los bines son en total 16, se puede observar claramente
que los valores permitidos de $x_t$ estan acotados al intervalo $(0, 1)$ y que los valores extremos --cercanos a cero y a la unidad-- son
m\'as frecuentes en la serie. Esto apoya nuestra conclusi\'on del apartado anterior, donde se ve que el espectro de Fourier, poseia 
una frecuencia m\'as intensa, mostrando una cuasi-periodicidad.

\begin{figure}[!h]
 \centering
 \includegraphics[width=0.5\textwidth,keepaspectratio=true]{../graficos/ej8c.png}
 % ej8c.png: 0x0 pixel, 300dpi, 0.00x0.00 cm, bb=
 \caption{Histograma de $x_t$}
 \label{fig:8c}
\end{figure}

\subsection{Apartado d}
\subsubsection{Enunciado}
Se pide obtener el llamado ``diagrama de \'orbitas'', en el plano $x$, $r$. Es necesario para esto realizar el calculo de los valores de 
$x_t$ para $3.4 < r < 4.$, utilizando adem\'as los valores correspondientes a $300 < t < 600$. Adem\'as se solicit\'o un c\'alculo extra, para
obtener un gr\'afico de la regi\'on ampliada correspondiente a $0.13 < x < 0.185$ y $3.847 < r < 3.856$.

\subsubsection{Soluci\'on}
Se program\'o el siguiente c\'odigo, que resolvia el calculo num\'erico del ejercicio para las \'orbitas:
\inputminted[firstline=16, linenos, firstnumber=1]{fortran}{../8/ej8d.f90}

Adem\'as se utiliz\'o para la ampliaci\'on el mismo c\'odigo, con modificaciones en las l\'{i}neas 15 y 16, donde
se definen los valores m\'aximos y m\'{i}nimos de $r$ permitidos para construir el diagrama de \'orbitas.

\subsubsection{Discusi\'on}
En la figura \ref{fig:8d_1} se pueden apreciar las \'orbitas para diferentes valores de $r$, 
y es f\'acil distinguir las bifurcaciones que se producen en las mismas para $r$ cada vez mayores.

Estas bifurcaciones llevan al sistema --en una llamada ``transcici\'on al caos''-- de manera paulatina
a una regi\'on del diagrama en el que $x_t$ se vuelve impredecible.

En la figura \ref{fig:8d_2} se aprecia la ampliaci\'on sobre la regi\'on dada en el enunciado, 
y se puede observar una estructura similar a la hallada para el diagrama de la figura \ref{fig:8d_1}, 
aunque no es totalmente id\'entica. Esto refleja lo charlado en clase, sobre la \textit{dimensi\'on fractal}
del diagrama de \'orbitas.

\begin{figure}[!h]
 \centering
 \includegraphics[width=0.9\textwidth,keepaspectratio=true]{../graficos/ej8d_1.png}
 % ej8d_1.png: 0x0 pixel, 300dpi, 0.00x0.00 cm, bb=
 \caption{\'Orbitas para $3.4 < r < 4.$.}
 \label{fig:8d_1}
\end{figure}

\begin{figure}[!h]
 \centering
 \includegraphics[width=0.9\textwidth,keepaspectratio=true]{../graficos/ej8d_2.png}
 % ej8d_1.png: 0x0 pixel, 300dpi, 0.00x0.00 cm, bb=
 \caption{\'Orbitas para $3.847 < r < 3.856$ y $0.13 < x_t < 0.185$.}
 \label{fig:8d_2}
\end{figure}

\subsection{Apartado e}
\subsubsection{Enunciado}
Se pide obtener el exponente de Lyapunov, para $2<r<4$.

\subsubsection{Soluci\'on}
Se program\'o el siguiente c\'odigo, para el c\'alculo solicitado:
\inputminted[firstline=16, linenos, firstnumber=1]{fortran}{../8/ej8e.f90}

\subsubsection{Discusi\'on}
En la figura \ref{fig:ej8e} se muestran los valores de Exponente de Lyapunov obtenidos.
Si bien se solicitaba recorrer los valores de $r$ entre $2$ y $4$, unicamente se pudieron calcular los 
exponentes para $3.55 < r < 4.$.

\begin{figure}[!h]
 \centering
 \includegraphics[width=0.7\textwidth,keepaspectratio=true]{../graficos/ej8e.png}
 % ej8e.png: 0x0 pixel, 300dpi, 0.00x0.00 cm, bb=
 \caption{Exponente de Lyapunov}
 \label{fig:ej8e}
\end{figure}

\section{Ejercicio 9: Oscilador de Pullen-Edmonds}
Para el ejercicio 9, se debe considerar el siguiente Hamiltoniano:
\begin{displaymath}
 H = \frac{1}{2}(p_1^2 + p_2^2) + \frac{1}{2}(q_1^2 + q_2^2) + \alpha q_1^2 q_2^2
\end{displaymath}
con $\alpha$ pertenecientes a los reales y los valores de masa $m$ y frecuencia angular
$\omega$ sean iguales a 1.


\subsection{Apartado a}
\subsubsection{Enunciado}
Se solicita integrar num\'ericamente las ecuaciones de movimiento utilizando un algoritmo
de Runge-Kutta de orden 4. El valor de $\alpha$ debe ser de $0.05$, y se deben elegir valores de energ\'{i}a 
$E = 5, 20, 100$. 
Se pide graficar valores de las coordenadas y momentos generalizados, con respecto al tiempo.
Las condiciones iniciales deben ser tales que $q_1 = 2$, $p_1 = 0$, $q_2=0$, $p_2=\sqrt{2E-4}$.

\subsubsection{Soluci\'on}
Se program\'o el siguiente c\'odigo, que resolvia el calculo num\'erico del ejercicio:
\inputminted[firstline=16, linenos, firstnumber=1]{fortran}{../9/ej9a.f90}
En la l\'{i}nea 14 se establece el valor de la energ\'{i}a del sistema, y en la siguiente
las condiciones iniciales.
El valor \'optimo de $h$ es aproximadamente $10^{-3}$, por lo que esta declarado como un par\'ametro.
Se utiliz\'o la funci\'on $f$ declarada en las l\'{i}neas 41-51 para poder calcular la integraci\'on de Runge-Kutta.
Esta funci\'on provino de linealizar el sistema mediante las ecuaciones de movimiento:
\begin{displaymath}
 \frac{\partial q_i}{\partial t} = \frac{\partial H}{\partial p_i} 
\end{displaymath}
\begin{displaymath}
 \frac{\partial p_i}{\partial t} = - \frac{\partial H}{\partial q_i}
\end{displaymath}

Lo cual entrega para nuestro Hamiltoniano:
\begin{align*}
 \dot{q_1} &= p_1  &   \dot{p_1} &= -q_1 -\alpha q_1 q_2^2 \\
 \dot{q_2} &= p_2  &   \dot{p_2} &= -q_2 -\alpha q_2 q_1^2
\end{align*}
Dando entonces la forma de la funci\'on $f$.

\subsubsection{Discusi\'on}
En la figura \ref{fig:9pq} se pueden apreciar los valores de las coordenadas y momentos generalizados en funci\'on
del tiempo para los valores de $0 < t < 200$. En este gr\'afico se desplazaron los valores reales de los osciladores
de energ\'ias $20$ y $100$ para poder apreciar mejor la evoluci\'on temporal de cada integraci\'on.

Es notable como el oscilador de alta energ\'{i}a posee frecuencias arm\'onicas que lo perturban del caso de energ\'{i}as
relativamente bajas.

\begin{figure}[!h]
 \centering
 \includegraphics[width=0.9\textwidth,keepaspectratio=true]{../graficos/ej9a_pq.png}
 % ej9a_pq.png: 0x0 pixel, 300dpi, 0.00x0.00 cm, bb=
 \caption{Coordenadas y momentos generalizados en funcion del tiempo. 
 Notar el desplazamiento artificial agregado para diferentes energ\'{i}as, 
 ya que en realidad cada oscilador oscila en el punto de equilibrio $(q_1, q_2)= (0, 0)$. 
 Esto ha sido agregado para poder distinguir las trayectorias de cada una de las fases.}
 \label{fig:9pq}
\end{figure}

\subsection{Apartado a}
\subsubsection{Enunciado}
Se solicita trazar las secciones de Poincar\'e en el plano $q_1$, $p_1$ para los valores de energ\'{i}a dados en el punto 
anterior. Las condiciones del plano son: $q_2 = 0$ y $p_2 > 0$. 

\subsubsection{Soluci\'on}
Se program\'o el c\'odigo num\'erico a continuaci\'on:
\inputminted[firstline=16, linenos, firstnumber=1]{fortran}{../9/ej9b.f90}

El cual es basicamente el mismo c\'odigo que en el apartado anterior, con la \'unica diferencia de 
las l\'{i}neas 23-28 donde se barre para $11\times 11 = 121$ condiciones iniciales diferentes.

\subsubsection{Discusi\'on}
En la figura \ref{fig:ej9b5} se aprecia el diagrama de secci\'on de Poincar\'e para la energ\'{i}a m\'as baja, 
mostrando la presencia de tres centros (sobre el eje de $q_1 = 0$ alrededor de los cuales las soluciones oscilan.
Esto quiere decir que las soluciones ser\'an peri\'odicas al menos en este plano del espacio de las fases.

Es f\'acil notar que los centros corresponden a puntos donde las soluciones ser\'{i}an est\'aticas, aunque 
no es posible tan s\'olo con estos diagramas, analizar si esos puntos son de equilibrio inestable o estable.

\begin{figure}[H]
 \centering
 \includegraphics[width=0.7\textwidth,keepaspectratio=true]{../graficos/ej9b_e5.png}
 % ej9b_e5.png: 0x0 pixel, 300dpi, 0.00x0.00 cm, bb=
 \caption{Secci\'on de Poincar\'e de $q_1$, $p_1$para $E=5$}
 \label{fig:ej9b5}
\end{figure}

En la figura \ref{fig:ej9b20} se puede observar el diagrama de secci\'on de Poincar\'e, pero para una 
energ\'{i}a mayor ($E=20$). En este caso, se pueden ver zonas donde las separatrices de los tres centros estables
se cruzan --en $(q, p) = (0, \pm 4)$ --. En estas regiones el comportamiento no esta claro, aunque es posible
decir que no conforman centros atractores. Por otro lado existen regiones de clara periodicidad, una vez m\'as, cerca
de los centros sobre el eje $q_2 = 0$.

\begin{figure}[H]
 \centering
 \includegraphics[width=0.7\textwidth,keepaspectratio=true]{../graficos/ej9b_e20.png}
 % ej9b_e5.png: 0x0 pixel, 300dpi, 0.00x0.00 cm, bb=
 \caption{Secci\'on de Poincar\'e de $q_1$, $p_1$para $E=20$}
 \label{fig:ej9b20}
\end{figure}

En la figura \ref{fig:ej9b100} se muestra el diagrama correspondiente a $E=100$. Se puede apreciar muy poca estructura, 
lo que nos dice que el comportamiento de este oscilador es un comportamiento aperi\'odico --al menos en los intervalos de 
tiempo que estamos integrando-- ya que es dificil asegurar que haya una concentraci\'on de puntos en el diagrama.
Esto quiere decirnos que en este corte del espacio de las fases, el oscilador no vuelve a retornar sobre configuraciones
antes visitadas (digamos combinaciones de $q_1$ y $p_1$), y que por lo tanto se mueve por toda la regi\'on de este espacio
que le son permitidas --existen restricciones debido a la energ\'{i}a del sistema.

\begin{figure}[H]
 \centering
 \includegraphics[width=0.7\textwidth,keepaspectratio=true]{../graficos/ej9b_e100.png}
 % ej9b_e5.png: 0x0 pixel, 300dpi, 0.00x0.00 cm, bb=
 \caption{Secci\'on de Poincar\'e de $q_1$, $p_1$para $E=100$}
 \label{fig:ej9b100}
\end{figure}



\end{document}          
